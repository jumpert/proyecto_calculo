\documentclass{article}

% Set page size and margins
% Replace `letterpaper' with `a4paper' for UK/EU standard size
\usepackage[letterpaper,top=2cm,bottom=2cm,left=3cm,right=3cm,marginparwidth=1.75cm]{geometry}
\usepackage[spanish]{babel}
\usepackage{url}
\usepackage{hyperref}
\usepackage{csquotes}
\usepackage{amsmath}
\usepackage{amssymb}
\usepackage{graphicx}
\usepackage{listings}
\usepackage{xcolor}
\usepackage{setspace}
\usepackage{float}

\graphicspath{ {assets/} }

%Code colors:
\definecolor{codegreen}{rgb}{0,0.6,0}
\definecolor{codegray}{rgb}{0.5,0.5,0.5}
\definecolor{codepurple}{rgb}{0.58,0,0.82}
\definecolor{backcolour}{rgb}{0.95,0.95,0.92}
\doublespacing

\lstdefinestyle{mystyle}{
    backgroundcolor=\color{backcolour},
    commentstyle=\color{codegreen},
    keywordstyle=\color{magenta},
    numberstyle=\tiny\color{codegray},
    stringstyle=\color{codepurple},
    basicstyle=\ttfamily\footnotesize,
    breakatwhitespace=false,
    breaklines=false,
    captionpos=b,
    keepspaces=true,
    numbers=left,
    numbersep=5pt,
    showstringspaces=false,
    showlines=false,
    showtabs=false,
    tabsize=2
}

\lstset{style=mystyle}

\NewDocumentCommand{\codeword}{v}{%
    \texttt{\textcolor{black}{#1}}%
}

\begin{document}

    \begin{titlepage}
        \centering
        {\includegraphics[width=0.5\textwidth]{logo2}\par}
        {\bfseries\LARGE Universidad Católica del Uruguay \par}
        \vspace{0.3cm}
        {\scshape\Large Facultad de Ingeniería \par}
        \vspace{0.3cm}
        {\scshape\Huge Proyecto - Polinomios de Taylor\par}
        \vspace{1cm}
        {\Large Cálculo Aplicado \par}
        {\Large Profesores: Maglis Mujica y Martín Perciante \par}
        \vfill
        {\Large Autores: \par}
        {\Large Luis Balduini (4.001.184-9)\\Leandro Casaretto (x.xxx.xxx-x)\\Juan Manuel Pérez (4.673.899-0) \par}
        \vfill
        {\Large \today \par}
    \end{titlepage}

    \section{Introducción}\label{sec:intro}
El presente informe aborda el uso del \textbf{polinomio de Taylor} como
herramienta para aproximar modelos físicos y simplificar la resolución de
problemas que, de otro modo, requerirían técnicas analíticas avanzadas.Se consideran dos situaciones independientes:

\begin{itemize}
\item \textbf{Parte 1:} la variación de la aceleración gravitatoria con la
distancia al centro terrestre, basada en la ley de gravitación universal.
\item \textbf{Parte 2:} la caída libre vertical con \emph{rozamiento lineal}
(fuerza de arrastre proporcional a la velocidad).
\end{itemize}

Ambos casos comparten el objetivo de responder: \emph{¿en qué régimen es
válido reemplazar la función exacta por un polinomio de baja orden y qué
error se comete?}

% --------------------------------------------------------------------
\section{Marco Teórico}\label{sec:teoria}

\subsection{Polinomio de Taylor}
Sea $f$ una función con derivadas hasta orden $n$ en un entorno de $a$.
El desarrollo de Taylor de orden $n$ es



Para $n=1$ y $n=2$ se obtienen la aproximación lineal y cuadrática,
respectivamente.

\subsection{Gravitación Universal}
La aceleración sobre un cuerpo de masa $m$ a distancia $r$ del centro
de la Tierra (masa $M$) es $a(r)=-\dfrac{GM}{r^2}$, donde $G$ es la
constante de gravitación universal.

\subsection{Rozamiento lineal}
Un cuerpo de masa $m$ sometido a un rozamiento $-b v$ obedece

Definiendo $\gamma=b/m$, la solución analítica para la posición es


% --------------------------------------------------------------------
\section{Desarrollo}\label{sec:desarrollo}

\subsection*{Parte 1 – Gravedad}
Los siete ítems se resuelven en orden:
\begin{enumerate}
\item Se deriva $f(r)=-\dfrac{GM}{r^2}$ y se presenta el Taylor de orden 1.
\item Se evalúa $f(R_T+h)$ para $h=8849,\text{m}$ y se calcula el error relativo.
\item Se construye el Taylor de orden 2 y se compara con el valor exacto.
\item Se grafica $f(r)$ junto a ambos polinomios en el entorno de $R_T$.
\item Se determina la distancia para que $|a|$ sea 1 % menor.
\item Análisis en radios pequeños (0.01 m  0.02 m) con
órdenes 2 y 3.
\item Gráfico ilustrativo.
\end{enumerate}
Las figuras y los scripts se incluyen en el repositorio adjunto
(\codeword{parte1_gravedad.py}).

\subsection*{Parte 2 – Caída con rozamiento}
\begin{enumerate}
\item Desarrollo de Taylor O(2) de $y(t)$; se discute la aceleración
inicial y el caso $\gamma\to0$.
\item Simulación numérica y superposición de curvas para los tres $\gamma$
(hormiga, persona, auto) y las seis combinaciones .
\item Obtención analítica de la velocidad terminal
$v_T=-\dfrac{g}{\gamma}$ y verificación en las simulaciones.
\item Criterio basado en $\gamma t\ll1$ para descartar la resistencia.
\end{enumerate}
Los resultados se comentan sobre cada gráfico generado por
\codeword{caida_con_rozamiento.py}.

% --------------------------------------------------------------------
\section{Conclusiones}\label{sec:conclusiones}
\begin{itemize}
\item Los polinomios de Taylor orden 1 y 2 son suficientes para aproximar
la ley $,-GM/r^2$ a menos de 0.3 % en la baja atmósfera.
\item En el modelo con rozamiento lineal, el término adicional
$\tfrac12\gamma v_0 t^2$ delimita la validez de la aproximación.La condición $\gamma t<0.3$ (<10 % de corrección) resultó un umbral
práctico en todas las simulaciones.
\item El uso del Taylor permitió justificar algebraicamente cuándo las
curvas con y sin aire se separan, antes de siquiera integrar
numéricamente el sistema.
\end{itemize}

% --------------------------------------------------------------------
\section{Referencias}
\begin{enumerate}
\item Beer, F. P., Johnston, E. R., \emph{Mecánica vectorial para ingenieros}. McGraw‑Hill.
\item Serway, R.
\emph{Física para ciencias e ingeniería}. Cengage Learning.
\item Notas de curso de Cálculo Aplicado – UCU (2024).
\end{enumerate}

\end{document}